\documentclass{article}
\usepackage{amsmath}
\usepackage{amsfonts}
\usepackage{accents}
\usepackage{pgfplots}
\usepackage{filecontents}  


\begin{document}
\title{Faster Method for Stencil computation}
\author{Robbie Buxton}
\maketitle
I propose a method for increasing the speed of computating stencil. 
Currently, this method works on $3\times 3$ star stencils. To my understanding it is generalizable to N dimensions for any width for star shapes. \\ \\
For a given stencil
\newcommand{\stencilTop}{\alpha_{1}}
\newcommand{\stencilLeft}{\beta_{1}}
\newcommand{\stencilMiddle}{\gamma}
\newcommand{\stencilRight}{\beta_{2}}
\newcommand{\stencilBottom}{\alpha_{2}}
\[\begin{matrix} 
	& \stencilTop  & \\
	\stencilLeft  & \stencilMiddle  & \stencilRight  \\
	& \stencilBottom & 
\end{matrix}\] \\
\newcommand{\verticalBands}{V}
\newcommand{\horizontalBands}{H}
\newcommand{\sourceGrid}{S}
Applied to a grid $\sourceGrid \in \mathbb{R}^{n \times n}$ $t$ times,
\cite{10.1145/3524059.3532392} shows that you can represent this the sum of two matrix multiplications
\[ \sourceGrid_t = \verticalBands \sourceGrid_{t-1} + \sourceGrid_{t-1} \horizontalBands, \> \> S_0 = S\]
where V and H are tridiagonal Toeplitz matrices. 
\[ \verticalBands = 
\begin{bmatrix}
	\frac{\stencilMiddle}{2} & \stencilBottom & & & & & & O \\ 
	\stencilTop & \frac{\stencilMiddle}{2} & \stencilBottom & & & & & \\
	& \stencilTop & . & . & & & & \\
	& &  . & . & . & & &  \\
	& & & . & . & . & & \\
	& & & & . & .& \stencilBottom \\
	O & & & & &  \stencilTop & \frac{\stencilMiddle}{2}
\end{bmatrix} \in \mathbb{R}^{n\times n} \]
\[\horizontalBands = 
\begin{bmatrix}
	\frac{\stencilMiddle}{2} & \stencilRight & & & & & & O \\ 
	\stencilLeft & \frac{\stencilMiddle}{2} & \stencilRight & & & & & \\
	& \stencilLeft & . & . & & & & \\
	& &  . & . & . & & &  \\
	& & & . & . & . & & \\
	& & & & . & .& \stencilRight \\
	O & & & & &  \stencilLeft & \frac{\stencilMiddle}{2}
\end{bmatrix} \in \mathbb{R}^{n\times n}\] \\ 
I believe this can be sped up by instead calculating
\newcommand{\innerSum}{C}
\newcommand{\verticalBasis}{{X_V}}
\newcommand{\verticalEigens}{{\Lambda_V}}
\newcommand{\horizontalBasis}{{X_H}}
\newcommand{\horizontalEigens}{{\Lambda_H}}
\newcommand{\transformedSourceGrid}{T}

\[ \sourceGrid_t = \verticalBasis \innerSum \horizontalBasis^{-1} \] 
Where
\[\verticalBasis \verticalEigens \verticalBasis^{-1} = \verticalBands \]
\[\horizontalBasis \horizontalEigens \horizontalBasis^{-1} = \horizontalBands\]
\[\transformedSourceGrid = \verticalBasis^{-1} \sourceGrid \horizontalBasis\]
\[\innerSum_{i,j} = \transformedSourceGrid_{i,j} (\verticalEigens_{i,i} + \horizontalEigens_{j,j})^n\]
Where $X_V, X_H$ are the eigenvectors and $\Lambda_A,\Lambda_B$ are the eigenvalues of $V,H$. \\ \\
Based on \cite{noschese2013tridiagonal} we can calculate the eigenvalues and vectors 
algebraically because of the tridiagonal Toeplitz form of $\verticalBands$ and $\horizontalBands$ 
\[ i,j = 1:n\]
\[ \verticalEigens_{i,i} = \frac{\stencilMiddle}{2} + 2\sqrt{\stencilTop \stencilBottom}\cos{\frac{i\pi}{n+1}}\]
\[ \verticalBasis_{i,j} = (\frac{\stencilTop}{\stencilBottom})^{\frac{j}{2}}\sin{\frac{ij\pi}{n+1}} \]
\[ \horizontalEigens_{j,j} = \frac{\stencilMiddle}{2} + 2\sqrt{\stencilLeft \stencilRight}\cos{\frac{j\pi}{n+1}} \]
\[ \verticalBasis_{i,j} = (\frac{\stencilLeft}{\stencilRight})^{\frac{j}{2}}\sin{\frac{ij\pi}{n+1}} \]
\subsection*{Complexity and speed}
I believe this algorithm has complexity $O(M)$ where M is the complexity of a matrix multiplication. 
The algorithm is effectively bound by the time required to compute 4 matrix multiplications (at worst $n^3$). The last two of which are highly likely to be sparse \\ \\
As $t$ gets large $C_{i,j} \rightarrow \in \{- \infty, -1, 0, 1, \infty\}$ which typically leads to a sparse matrix 
which results in a performance speed up as $t$ increases.\\ \\
The algorithm currently uses non-sparse BLAS libraries for the matrix multiplications which is at worst $O(n^3)$. 
I plan to switch to sparseBLAS libaries.  \\ \\
I believe calculating the innerSum $\innerSum$ is done in $O(n^2)$ as taking large exponents is effectively in constant time. 
This means the algorithm is invariant under $t$. \\ \\ 
Currently, calculating the eigenvalues and eigenvectors is done in $O(n^2)$ time using the algorithm from \cite{noschese2013tridiagonal}.
From a quick skim read of \cite{bogoya2022fast} I believe you can quickly calculate the eigenvalues and vectors of non tridiagonal Toeplitz 
matrices that would arise from stencils with a width greater than 3. \\ \\
For $N$ dimensions $n$ Toeplitz matrices need to be diagonalized, if the matrices are the same the result can be reused. 

\subsection*{Proof of correctness}
\begin{flalign*}
\sourceGrid_t &= \verticalBands \sourceGrid_{t-1} + \sourceGrid_{t-1} \horizontalBands, \> \> S_0 = S &&\\
&= \sum_{k = 0}^t {t \choose k} \verticalBands^{k} \sourceGrid \horizontalBands^{n-k} &&\text{By self substitution until base case}\\
&= \sum_{k = 0}^t {t \choose k} (\verticalBasis \verticalEigens \verticalBasis^{-1})^{k} \sourceGrid (\horizontalBasis \horizontalEigens \horizontalBasis^{-1})^{n-k} &&\text{Diagonalizing $V$ and $H$}\\
&= \sum_{k = 0}^t {t \choose k} \verticalBasis \verticalEigens^{k} \verticalBasis^{-1} \sourceGrid \horizontalBasis \horizontalEigens^{n-k} \horizontalBasis^{-1} &&\text{Bringing the powers inside the basis change}\\
&= \sum_{k = 0}^t {t \choose k} \verticalBasis \verticalEigens^{k} \transformedSourceGrid \horizontalEigens^{n-k} \horizontalBasis^{-1} &&\text{Substituting $\verticalBasis^{-1} \sourceGrid \horizontalBasis$ with $T$ }\\
&=   \verticalBasis (\sum_{k = 0}^t {t \choose k} \verticalEigens^{k} \transformedSourceGrid \horizontalEigens^{n-k}) \horizontalBasis^{-1} &&\text{Taking the basis change out of the sum}\\
\end{flalign*}
\begin{flalign*}
({\sum_{k = 0}^t {t \choose k} \verticalEigens^{k} \transformedSourceGrid \horizontalEigens^{n-k}})_{i,j} &= \sum_{k = 0}^t {t \choose k} ({\verticalEigens_{i,i}})^{k} (\transformedSourceGrid_{i,j}) ({\horizontalEigens_{j,j}})^{n-k} && \text{Using the properties of diagonal matrices}\\
&= (\transformedSourceGrid_{i,j})  ({\horizontalEigens_{j,j}})^{n} \sum_{k = 0}^t {t \choose k} (\frac{\verticalEigens_{i,i}}{\horizontalEigens_{j,j}})^{k} &&\text{Pulling out the constant factors}\\
&= (\transformedSourceGrid_{i,j})  ({\horizontalEigens_{j,j}})^{n} (1 + \frac{\verticalEigens_{i,i}}{\horizontalEigens_{j,j}})^{n} &&\text{Simplifying using the binomial theorem}\\
&= (\transformedSourceGrid_{i,j})  ({\horizontalEigens_{j,j}}+ \verticalEigens_{i,i})^{n} = C_{i,j} &&\text{Canceling}\\
\end{flalign*}
\begin{flalign*}
\sourceGrid_t &= \verticalBasis C \horizontalBasis^{-1} &&
\end{flalign*}
\subsection*{Test results}

\begin{filecontents*}{data1.csv}
	size, devito, custom
	100, 0.151409, 0.012753
	200, 0.171719, 0.024297
	300, 0.267383, 0.021378
	400, 0.326060, 0.025595
	500, 0.522454, 0.046935
	600, 1.061937, 0.053750
	700, 1.255377, 0.070553
	800, 1.755467, 0.075158
	900, 2.422379, 0.127462
	1000, 3.423538, 0.191081
	1100, 4.749781, 0.188261
	1200, 6.282515, 0.206073
	1300, 7.940687, 0.258702
	1400, 9.948888, 0.324063
	1500, 12.294626, 0.355924
	1600, 15.000360, 0.390331
	1700, 18.036925, 0.505403
	1800, 22.403344, 0.591835
	1900, 27.284942, 0.685276
	2000, 32.834737, 0.802248
	2100, 36.571196, 1.011154
	2200, 43.546261, 1.101451
	2300, 49.733833, 1.291154
	2400, 54.443949, 1.342048
	2500, 62.262221, 1.385411
	2600, 70.272917, 1.483700
	2700, 79.364802, 1.613500
	2800, 85.705118, 1.689242
	2900, 98.216481, 2.267311
	3000, 112.504909, 2.574438
	3100, 119.051002, 2.499289
	3200, 128.200307, 2.453369
	3300, 168.536610, 3.497994
	3400, 168.635349, 3.278498
	3500, 180.232800, 3.656672
\end{filecontents*}
Running on an i5-8250U 4 cores, 8 threads. 3.40 GHZ \\ \\
\begin{tikzpicture}
	\begin{axis} [ymode=log,ymax = 500,xmax = 3750, ylabel={time},xlabel={size and iterations},legend pos=south east]
	\addplot table [x=size,y=devito, col sep=comma] {data1.csv};
	\addplot table [x=size,y=custom, col sep=comma] {data1.csv};
	\addlegendentry{Devito}
	\addlegendentry{Custom}
	\end{axis}
\end{tikzpicture}

\begin{filecontents*}{data2.csv}
	iterations, devito, custom
	25, 0.392590, 0.790663
	50, 0.690924, 1.942880
	75, 1.017770, 2.500129
	100, 1.327902, 3.523172
	125, 1.667693, 3.869910
	150, 1.992307, 3.411621
	175, 2.312899, 2.935495
	200, 2.647703, 2.556436
	225, 2.970207, 2.337855
	250, 3.301973, 2.147661
	275, 3.640479, 1.791232
	300, 3.968999, 1.591348
	325, 4.299237, 1.548726
	350, 4.618517, 1.552998
	375, 4.931562, 1.533609
	400, 5.278637, 1.521532
	425, 5.593655, 1.412656
	450, 6.047294, 1.415830
	475, 6.714649, 1.302995
	500, 6.856577, 1.235695
	525, 7.029075, 1.207368
	550, 7.306775, 1.134019
	575, 7.673165, 1.067803
	600, 8.009989, 1.050725
	625, 8.313540, 0.992116
	650, 8.645910, 0.978663
	675, 8.955524, 1.001663
	700, 9.264035, 0.987331
	725, 9.576102, 0.975227
	750, 9.892396, 1.011991
	775, 10.209912, 1.014369
	800, 10.542156, 0.959242
	825, 10.965361, 0.992886
	850, 11.213993, 1.013995
	875, 11.521042, 0.978162
	900, 11.959000, 0.946707
	925, 12.275203, 0.923441
	950, 12.517497, 0.949448
	975, 12.787292, 0.898449
	1000, 13.123104, 0.853687
\end{filecontents*}
Running on an i5-9600K 6 cores, 6 threads. 4.60 GHZ \\ \\
\begin{tikzpicture}
	\begin{axis} [ymode=log,xmax = 1050, ylabel={time},xlabel={iterations},legend pos=north west]
	\addplot table [x=iterations,y=devito, col sep=comma] {data2.csv};
	\addplot table [x=iterations,y=custom, col sep=comma] {data2.csv};
	\addlegendentry{Devito}
	\addlegendentry{Custom}
	\end{axis}
\end{tikzpicture}

\begin{filecontents*}{data3.csv}
	iterations, max, average
	100, 0.000799, 0.000234
	200, 0.002205, 0.000294
	300, 0.001442, 0.000488
	400, 0.001979, 0.000399
	500, 0.002968, 0.001078
	600, 0.005603, 0.000918
	700, 0.006598, 0.001099
	800, 0.004888, 0.001594
	900, 0.003040, 0.000717
	1000, 0.008368, 0.001053
	1100, 0.005782, 0.002070
	1200, 0.009882, 0.004165
	1300, 0.006735, 0.003384
	1400, 0.007641, 0.002941
	1500, 0.012362, 0.002723
	1600, 0.011587, 0.002277
	1700, 0.010204, 0.002933
	1800, 0.008976, 0.002819
	1900, 0.009274, 0.002207
	2000, 0.012398, 0.002593
	2100, 0.008714, 0.002496
	2200, 0.014281, 0.004674
	2300, 0.014406, 0.004258
	2400, 0.014293, 0.004080
	2500, 0.016248, 0.003800
\end{filecontents*}
Running on an i5-8250U 4 cores, 8 threads. 3.40 GHZ \\ \\
\pgfplotsset{scaled y ticks=false}
\begin{tikzpicture}
	\begin{axis} [xmax = 2550, 
								ylabel={Percent error},
								xlabel={size and iterations},
								legend pos=north west,
								yticklabel style={/pgf/number format/fixed}
								]
	\addplot table [x=iterations,y=max, col sep=comma] {data3.csv};
	\addplot table [x=iterations,y=average, col sep=comma] {data3.csv};
	\addlegendentry{Max element error}
	\addlegendentry{Average element error}
	\end{axis}
\end{tikzpicture}

\subsection*{Future work}
If I continue working on this algorithm I am fairly confident I can generalize it to work on arbitrary sized star stencils in $N$ dimensions. 
I am unsure if this method would apply to blocked stencils. This is something I can possibly investigate after, but intuitively I expect it to be less efficient. 
\bibliographystyle{ieeetr}
\bibliography{citation} 
\end{document}
